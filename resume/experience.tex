\cvsection{Research Experience}

\begin{cventries}
\vspace{-0.1cm}
  \cventry
    {\textnormal{Advisor: Prof. Wen Zhao @ USTC}} % Job title
    {Model-independent test of the parity symmetry of gravity with gravitational waves} % Organization
    {\textcolor{awesome-emerald}{\textbf{09/2018 - 05/2019}}} % Location
    {} % Date(s)
    {
      \begin{cvitems} % Description(s) of tasks/responsibilities
        \item{\textnormal{Aimed to develop a waveform-independent method to examine the parity symmetry of gravity with the data of gravitational waves.}}
        \item{\textnormal{Extract left-hand and right-hand polarization modes of gravitational waves to measure the arrival time difference between the two polarizations, thus test the parity symmetry of gravity by constraining the velocity birefringence of gravitational waves.}}
        \item{\textnormal{Conducted tests on simulated data to prove the feasibility of this method.}}
        \item{\textnormal{Work on real data GW150914 and GW170817. Strongly constrained the scale of parity violation.}}
      \end{cvitems}
    }
  \vspace{0.4cm}
  \cventry
    {\textnormal{Advisor: Prof. Lingqin Wen @ University of Western Australia}} % Job title
    {Semianalytical approach for sky localization of gravitational waves} % Organization
    {\textcolor{awesome-emerald}{\textbf{07/2019 - 10/2021}}} % Location
    {} % Date(s)
    {
      \begin{cvitems} % Description(s) of tasks/responsibilities
        \item{\textnormal{Aimed to shorten the latency of sky localization of gravitational wave sources through Bayesian method to replace the current BAYESTAR method.}}
        \item{\textnormal{Conducted a Monte Carlo experiment and chose an Gaussian approximation of the true distribution of extrinsic parameters.}}
        \item {\textnormal{Innovated an analytical solution of the Bayesian posterior probability density for gravitational wave sources sky localization, which was a numerical solution in previous method.}}
        \item{\textnormal{Conducted various kinds of simulation tests, based on ideal power spectrum density and O2 power spectrum density, to check the self-consistency of my method through a P - P plot, which showed a very good result.}}
        \item {\textnormal{Qian Hu changed the approximation of the prior and improved my original method.}}
      \end{cvitems}
    }
  \vspace{-0.4cm}
\end{cventries}
